% Copyright 2009 by Henrik Kroeger and Sarah Paczkowski
%
% This file may be distributed and/or modified
% under the GNU Public License.

\documentclass[a4paper]{article}
\usepackage[ngerman]{babel}
\usepackage[T1]{fontenc}
\usepackage[utf8]{inputenc}

\usepackage{booktabs}
\usepackage{cellspace}
\usepackage{dcolumn}

\usepackage{amsmath}
\usepackage[amssymb]{SIunits}
\usepackage{sistyle}
\SIstyle{German}

\usepackage[loose]{subfigure}
\usepackage{tikz}
\usepackage{pgffor}
\usepackage{pgfmath}

\usepackage[square]{natbib}
\usepackage{hyperref}

\begin{document}
\bibliographystyle{geralpha}

\title{Versuche zum Quanten-Hall-Effekt}
\author{Sarah Paczkowski \and Henrik Kröger}
\date{Durchgeführt 27. und 28. Oktober 2009}

\maketitle

\noindent
Der \emph{Hall-Effekt} ist das Entstehen einer Potentialdifferenz
(der \emph{Hall-Spannung} $U_H$)
in einem elektrischen Leiter senkrecht zu einem elektrischen Strom $I$ und
senkrecht zu einem Magnetfeld $B$.
Der \emph{Hall-Widerstand} $R_H = \frac{U_H}{I}$,
verhält sich beim gewöhnlichen Hall-Effekt wie folgt:
\begin{equation} R_H = A_H \frac{B}{d} \end{equation}
$A_H$ ist dabei die materialabhängige \emph{Hall-Konstante} und
$d$ die Dicke des Leiters in Richtung des Magnetfeldes.
Im Fall eines Leiters mit nur einer Ladungsträgerart errechnet sich
die Hall-Konstante aus der Ladungsträgerdichte $n_e$ und
der Ladung $q$ der Ladungsträger:
\begin{equation} A_H = \frac{1}{n_eq} \end{equation}

\begin{figure}[hb]
\begin{center}
\begin{tikzpicture}

  % Box
  \draw (0,0) rectangle (2,1);
  \draw (0.2,0.2) rectangle (2.2,1.2);
  \draw (0,0) -- (0.2,0.2);
  \draw (2,1) -- (2.2,1.2);
  \draw (2,0) -- (2.2,0.2);
  \draw (0,1) -- (0.2,1.2);

  % Strom I
  \draw[->] (-1,0.6) -- (0.1,0.6) node[above,near start] {$I$};
  \draw[->] (2.1,0.6) -- (3,0.6) node[above,near end] {$I$};

  % Megnetfeld B
  \draw[->,thin] (1.3,0.8) -- (0.8,0.3) node[right,midway] {$B$};

  % Breite d
  \draw (0,1.25) -- (0.2,1.45) node[right,above] {$d$};
  \draw[very thin] (0,1) -- (0,1.3);
  \draw[very thin] (0.2,1.2) -- (0.2,1.5);

  % Voltmeter U
  \draw (4,0.6) node [circle, draw] (Voltmeter) {V} node[right=4mm] {$U_H$};
  \draw [out=90,in=90] (1,1.1) to (Voltmeter);
  \draw [out=-90,in=-90] (1,0.1) to (Voltmeter);

\end{tikzpicture}
\end{center}
\caption{Zum klassischen Hall-Effekt}
\end{figure}

Bei hohen Magnetfeldern (mehrere Tesla) und
tiefen Temperaturen (wenige Kelvin)
setzt dagegen der \emph{Quanten-Hall-Effekt} ein.
Hier hängt die Hall-Spannung nicht mehr linear vom Magnetfeld ab,
sondern weist Stufen auf.
Der Hall-Widerstand kommt dabei in Brüchen
der \emph{von-Klitzing-Konstanten} $R_K$ vor:
\begin{equation} R_H = \frac{R_K}{\nu} \label{eq:hallwiderstand} \end{equation}
Dabei ist $\nu$ der sogenannte \emph{Füllfaktor}.
Bei ganzzahligen Füllfaktoren spricht man
vom \emph{integralen Quanten-Hall-Effekt},
bei gebrochen-rationalen
vom \emph{fraktionalen Quanten-Hall-Effekt}.
Für den letzteren braucht es noch tiefere Temperaturen
und eine geringere \emph{Elektronenkonzentration} $n_e$ im Leiter.

Die von-Klitzing-Konstante lässt sich aus Naturkonstanten errechnen:
\begin{equation}
R_K = \frac{h}{e^2} \approx \SI{25,812807}{k\Omega}
\end{equation}


\section*{Proben}

Als Leiter für die Versuche werden
Aluminium-Gallium-Arsenid-Heterostrukturen verwendet.
Diese werden in $\SI{3}{mm}\times\SI{3}{mm}$ großen Stücken
aus einem Viertelwafer herausgebrochen.

Mithilfe der UV-Lithographie werden
die Proben mit der gewünschten Struktur versehen.
Dazu wird zunächst Fotolack aufgeschleudert.
Teile des Probenmaterials werden anschließend durch Ätzen entfernt
und es entsteht die sogenannte Mesa.

Nun wird Germanium, Gold und Nickel auf die Probe aufgedampft und
der Lack entfernt.
Des Weiteren werden die elektrischen Kontakte einlegiert.

Bei Probe 5 wird zusätzlich ein \emph{Topgate} durch Lithographie strukturiert.
Die Topgate-Kontakte sind Schottky-Kontakte,
da sie nicht einlegiert werden.

Beide Proben werden auf einen Probenhalter aufgebracht und
die Kontakte der Probe mit denen
des Probenhalters verbunden (siehe Abbildung~\ref{fig:bondschemata}).

\begin{figure}[hb]
\begin{center}
\scriptsize
\subfigure[Probe 4, ohne Topgate]{
\begin{tikzpicture}[%
scale=0.2,
inner/.style={rectangle,draw,minimum size=2mm},
outer/.style={rectangle,draw}
]
  % Innere Kontakte
  \draw (0,0) node[inner] (inner 11) {};
  \draw (0,-9) node[inner] (inner 5) {};

  \draw (-1.5,-3) node[inner] (inner 8) {};
  \draw (-1.5,-4.5) node[inner] (inner 7) {};
  \draw (-1.5,-6) node[inner] (inner 6) {};
  \draw (-1.5,-7.5) node[inner] {};

  \draw (1.5,-1.5) node[inner] {};
  \draw (1.5,-3) node[inner] (inner 13) {};
  \draw (1.5,-4.5) node[inner] (inner 19) {};
  \draw (1.5,-6) node[inner] (inner 4) {};

  % Carrier
%  \draw (-5,-13) rectangle +(17,17);
%  \draw (-7,-15) rectangle +(21,21);
  \foreach \x in {1,...,5}
    \pgfmathparse{3*(5-\x)-1.75}
    \draw (\pgfmathresult,-15) node[outer,anchor=south east,minimum height=0.4cm] (outer \x) {0\x};
  \foreach \x in {11,...,15}
    \pgfmathparse{3*(\x-11)-1.75}
    \draw (\pgfmathresult,4) node[outer,anchor=south east,minimum height=0.4cm] (outer \x) {\x};
  \foreach \x in {6,...,9}
    \pgfmathparse{2*(\x-6)-9}
    \draw (-5,\pgfmathresult) node[outer,anchor=south east,minimum width=0.4cm] (outer \x) {0\x};
  \draw (-5,-1) node[outer,anchor=south east,minimum width=0.4cm] (outer 10) {10};
  \foreach \x in {16,...,20}
    \pgfmathparse{2*(16-\x)-1}
    \draw (14,\pgfmathresult) node[outer,anchor=south east,minimum width=0.4cm] (outer \x) {\x};

  % Verbindungen
  \draw[in=-90,out=90] (outer 5) to (inner 5);
  \draw[in=0,out=90] (outer 4) to (inner 4);
  \draw[in=90,out=-90] (outer 11) to (inner 11);
  \draw[in=0,out=-90] (outer 13) to (inner 13);
  \draw[in=180,out=0] (outer 8) to (inner 8);
  \draw[in=180,out=0] (outer 7) to (inner 7);
  \draw[in=180,out=0] (outer 6) to (inner 6);
  \draw[in=0,out=180] (outer 19) to (inner 19);

\end{tikzpicture}
}
\qquad
\subfigure[Probe 5, mit Topgate]{
\begin{tikzpicture}[%
scale=0.2,
inner/.style={rectangle,draw,minimum size=2mm},
outer/.style={rectangle,draw}
]

  % Innere Kontakte
  \draw (0,0) node[inner] (inner 12) {};
  \draw (0,-9) node[inner] (inner 4) {};

  \draw (-1.5,-1.5) node[inner] {};
  \draw (-1.5,-3) node[inner] (inner 9) {};
  \draw (-1.5,-4.5) node[inner] (inner 8) {};
  \draw (-1.5,-6) node[inner] (inner 7) {};
  \draw (-1.5,-7.5) node[inner] (inner 5) {};

  \draw (1.5,-1.5) node[inner] (inner 13) {};
  \draw (1.5,-3) node[inner] (inner 17) {};
  \draw (1.5,-4.5) node[inner] (inner 18) {};
  \draw (1.5,-6) node[inner] (inner 3) {};

  % Carrier
%  \draw (-5,-13) rectangle +(17,17);
%  \draw (-7,-15) rectangle +(21,21);
  \foreach \x in {1,...,5}
    \pgfmathparse{3*(5-\x)-1.75}
    \draw (\pgfmathresult,-15) node[outer,anchor=south east,minimum height=0.4cm] (outer \x) {0\x};
  \foreach \x in {11,...,15}
    \pgfmathparse{3*(\x-11)-1.75}
    \draw (\pgfmathresult,4) node[outer,anchor=south east,minimum height=0.4cm] (outer \x) {\x};
  \foreach \x in {6,...,9}
    \pgfmathparse{2*(\x-6)-9}
    \draw (-5,\pgfmathresult) node[outer,anchor=south east,minimum width=0.4cm] (outer \x) {0\x};
  \draw (-5,-1) node[outer,anchor=south east,minimum width=0.4cm] (outer 10) {10};
  \foreach \x in {16,...,20}
    \pgfmathparse{2*(16-\x)-1}
    \draw (14,\pgfmathresult) node[outer,anchor=south east,minimum width=0.4cm] (outer \x) {\x};

  % Verbindungen
  \draw[in=180,out=0] (outer 7) to (inner 7);
  \draw[in=180,out=0] (outer 8) to (inner 8);
  \draw[in=180,out=0] (outer 9) to (inner 9);
  \draw[in=0,out=180] (outer 17) to (inner 17);
  \draw[in=0,out=180] (outer 18) to (inner 18);
  \draw[in=90,out=-90] (outer 12) to (inner 12);
  \draw[in=0,out=-90] (outer 13) to (inner 13);
  \draw[in=0,out=90] (outer 3) to (inner 3);
  \draw[in=-90,out=90] (outer 4) to (inner 4);
  \draw[in=180,out=90] (outer 5) to (inner 5);

\end{tikzpicture}
}
\end{center}
\normalsize
\caption{Bondschemata}
\label{fig:bondschemata}
\end{figure}

\pagebreak
\section*{Magnetotransportmessung}
Im Folgenden sollen die Eigenschaften der Probe ermittelt werden.
Dafür werden Hallwiderstand und Längswiderstand in
Abhängigkeit vom Magnetfeld ermittelt.

Probe 4 wird in den Magnetoprobenstab und
dieser in die Heliumkanne eingebaut.
Für jede der Messungen wird das Magnetfeld von $\SI{0}{T}$ bis $\SI{5}{T}$
hoch- und wieder heruntergefahren.
Zwischen den Kontakten 5 und 11 fließt ein Strom von $\SI{10}{\mu A}$.

In einer ersten Messung wurde die Hallspannung $U_{xy}$ an den
Kontakten 8 und 13 gemessen, die Längsspannung $U_{xx}$ an 6 und 8.
Bei der zweiten Messung kamen die Kontakte 4 und 6 für
die Hallspannung zum Einsatz, an 4 und 13 wurde die Längsspannung gemessen.
Bei der dritten Messung wurde über die Kontakte 7 und 8 die Längsspannung gemessen.

In Abbildung~\ref{fig:magnetotransport} sind die entsprechenden Hall- und Längswiderstände
aus den gemessenen Spannungen dargestellt ($ R_i = \frac{U_i}{\SI{10}{\mu A}} $).
Die Messwerte vom Hoch- und Herunterfahren des Magnetfeldes
liegen dabei aufeinander, weswegen nur Messwerte des Hochfahrens dargestellt werden.

\begin{figure}[hb]
    \input{Magnetotransport_1}
    \caption{
	Hall- und Längswiderstände an Probe 4 (kein Topgate)
	während \SI{10}{\micro{}A} zwischen den Kontakten 5 und 11 fließen.}
  \label{fig:magnetotransport}
\end{figure}


\subsection*{Elektronenkonzentration}
Mit den durchgeführten Messungen kann nun die Elektronenkonzentration $n_e$
in der Probe bestimmt werden. Das soll im Folgenden auf
drei verschiedene Arten geschehen.

\subsubsection*{Bestimmung über den Füllfaktor}
Formel~(2.30) \citet{Bockhorn}
ermöglicht die Berechnung der Elektronenkonzentration
über den Füllfaktor~$\nu$:
\begin{equation} n_e = \frac{\nu\, B(\nu)\, e}{h} \end{equation}
Dafür wird der Füllfaktor~$\nu$ aus den Hallplateaus von
Abbildung~\ref{fig:magnetotransport} mit Formel~\eqref{eq:hallwiderstand}
bestimmt.
$B(\nu)$ bezieht sich auf die Mitte der Hallplateaus,
welche am genausten über die Minima
der Schubnikow-de-Haas-Oszillationen\footnote{deutsche Transliteration}
abgelesen werden.

In Tabelle~\ref{tbl:elektronen_fuellfaktor} finden sich
die Ergebnisse für die ablesbaren Füllfaktoren.
Wegen zu hoher Temperaturen können nur zu
geraden Füllfaktoren Hallplateaus gefunden werden.

\begin{table}[h!]
\begin{center}
\begin{tabular}{rcccc}
\toprule
$\nu$ &
$\frac{\frac{R_K}{\nu}}{\SI{}{\Omega}}$ &
$\frac{R^\text{mess}_{xy}}{\SI{}{\Omega}}$ &
$\frac{B}{\SI{}{T}}$ & 
$\frac{n_e}{\SI{e15}{m^{-2}}}$ \\
\midrule
4 & 6454 & 6349 & 4,79 & \num{4.632} \\
6 & 4303 & 4300 & 3,21 & \num{4.656} \\
8 & 3227 & 3227 & 2,41 & \num{4.661} \\
10& 2581 & 2582 & 1,92 & \num{4.642} \\
\bottomrule
\end{tabular}
\end{center}
\caption{Elektronenkonzentrationen bestimmt über den Füllfaktor}
\label{tbl:elektronen_fuellfaktor}
\end{table}


\pagebreak

\subsubsection*{Bestimmung über $B^{-1}$-Oszillation}
Mit Formel (2.31) \citet{Bockhorn}
kann man die Elektronenkonzentration aus der
Oszillation~$\Delta\,B^{-1}$
der Längsspannung~$R_{xx}$ in Abhängigkeit vom
reziproken Magnetfeld~$B^{-1}$ berechnen:
\begin{equation}
n_e = \frac{2e}{h}\left(\Delta B^{-1}\right)^{-1}
\label{eq:elektronen_osz}
\end{equation}
Die Oszillation selbst ist in Abbildung~\ref{fig:elektronen_osz}
dargestellt.
Die Maxima und daraus berechnet die Oszillationen
sind in Tabelle~\ref{tbl:elektronen_osz} zu finden.
Aus dem Mittelwert $\SI{0.1065}{T^{-1}}$ ergibt sich
die Elektronenkonzentration:
\begin{equation*} n_e = \SI{4,5404e15}{m^{-2}} \end{equation*}

\begin{figure}[h!]
\begin{center}
\input{Elektronenkonzentration_1.tex}
\end{center}
\caption{Längsspannung~$R_{xx}$ in Abhängigkeit vom
reziproken Magnetfeld}
\label{fig:elektronen_osz}
\end{figure}

\begin{table}[h!]
\begin{center}
\begin{tabular}{cc}
\toprule
$\frac{B^{-1}}{\SI{}{T^{-1}}}$ &
$\frac{\Delta B^{-1}}{\SI{}{T^{-1}}}$ \\
\midrule
0,2448 & —\\
0,3580 & 0,1132 \\
0,4642 & 0,1061 \\
0,5701 & 0,1059 \\
0,6763 & 0,1061 \\
0,7801 & 0,1038 \\
0,8839 & 0,1038 \\
\bottomrule
& 0,1065
\end{tabular}
\end{center}
\caption{Maxima der $B^{-1}$-Oszillation aus
Abbildung~\ref{fig:elektronen_osz}
und
entsprechende Periodendauer~$\Delta\,B^{-1}$,
sowie Mittelwert}
\label{tbl:elektronen_osz}
\end{table}


\subsubsection*{Bestimmung durch die Steigung des klassischen Hall-Effekts}
Nach \citet{Bockhorn} kann mit Formel (2.32) die Elektronenkonzentration
auch aus der Steigung des Hallwiderstandes $R_{xy}$
(in Abhängigkeit des Magnetfeldes $B$) im Bereich des klassischen Halleffektes
bestimmt werden:
\begin{equation}
n_e = \frac{1}{e}\left(\frac{\partial R_{xy}}{\partial B}\right)^{-1}
\label{eq:ele_konz_steigung}
\end{equation}

Der Bereich des klassischen Hall-Effekts wird hier als dieser angenommen, wo
das Magnetfeld nicht größer als $\SI{1}{T}$ ist.
In diesem (siehe Abbildung \ref{fig:magnetotransport}) beträgt die Steigung:
\begin{equation*}
\frac{\partial R_{xy}}{\partial B} = \SI{1350,34}{\Omega/T}
\end{equation*}
Woraus sich folgende Elektronenkonzentration berechnet:
\begin{equation*} n_e = \SI{4.6227e15}{m^{-2}} \end{equation*}



\subsection*{Elektronenbeweglichkeit}
Nach Formel (2.33) \citet{Bockhorn} berechnet sich
die Elektronenbeweglichkeit~$\mu_e$ aus
der Elektronenkonzentration~$n_e$ und
dem spezifischen Widerstand~$\rho_0$:
\begin{equation}
\mu_e = \frac{1}{\rho_0 \, e \, n_e}
\end{equation}
Der spezifische Widerstand wiederum lässt sich so angeben:
\begin{equation}
\rho_0 = \frac{R_{xx}(B=0) \, b}{l}
\end{equation}
Damit:
\begin{equation}
\label{eq:elektronenbeweglichkeit}
\mu_e = \frac{l}{R_{xx}(B=0)\, b \, e \, n_e}
\end{equation}
Einsetzen des Mittelwerts der Elektronenkonzentrationen
\begin{equation*}
\overline n_e = \SI{4.6257e15}{m^{-2}}
\end{equation*}
und des Längswiderstandes $R_{xx}$ bei $B=0$ gemessen zwischen den
Kontakten 6 und 8 (erste Messung, $l=\SI{550}{\mu m}$)
\begin{equation*}
R_{xx}(B=0) = \SI{2.03354E2}{\Omega}
\end{equation*}
liefert:
\begin{equation*}
\mu_e = \frac{ \SI{550}{\mu m} }
 	     {\SI{2.03354E2}{\Omega} \cdot \SI{200}{\mu m} \cdot e \cdot \SI{4.6257e15}{m^{-2}}}
      = \SI{18.249}{m^2 / Vs}
\end{equation*}


\subsection*{Unterschiede zwischen den Kontaktpaaren}

Wie in Abbildung \ref{fig:magnetotransport} zu sehen ist,
gibt es keine nennenswerten Unterschiede zwischen den Kontaktpaaren.
Lediglich liegen die Kontakte 6 und 8, sowie 4 und 13
doppelt so weit auseinander,
wie die Kontakte 7 und 8, wodurch die Amplitude der
Schubnikow-de-Haas-Oszillation erhöht ist.

Die Elektronenbeweglichkeit beim Kontaktpaar 7 und 8 ist
nach Formel \eqref{eq:elektronenbeweglichkeit}:
\begin{equation*}
\mu_e = \frac{ \SI{275}{\mu m} }
 	     {\SI{1.04212e2}{\Omega} \cdot \SI{200}{\mu m} \cdot e \cdot \SI{4.6257e15}{m^{-2}}}
      = \SI{17.805}{m^2 / Vs}
\end{equation*}
Das ist ein Unterschied von \SI{2.4}{\%}.
Der Quanten-Hall-Effekt ist schließlich unabhängig von der Probengeometrie.

\subsection*{keine fraktionalen Füllfaktoren}
Es wurden keine fraktionalen Füllfaktoren gemessen,
da die Elektronenbeweglichkeit $\mu_e$ mit $\SI{18.249}{m^2 / Vs}$
zu gering ist.
Es ist weiterhin mit dem verwendeten flüssigen Helium nicht kalt genug.

\newpage

\section*{Dauerhafter Photoeffekt}
Durch Beleuchtung der Probe mit einer LED gelangen Elektronen
aus dem Gallium-Arsenid-Substrat in das zweidimensionale Elektronengas.
Da die Probe auf unter $\SI{70}{K}$ abgekühlt wurde,
bleibt dieser Photostrom auch nach dem Ausschalten der LED erhalten.
Dadurch erhöht sich die Elektronenkonzentration in der Probe.

Die Messung wurde wieder mit Probe 4 und einem Strom von $\SI{10}{\mu A}$
zwischen den Kontakten 5 und 11 durchgeführt.
Vor der Messung wurde die Probe kurz durch die LED beleuchtet.
In der ersten Messung wurde zweimal die Längsspannung~$U_{xx}$ gemessen und
zwar zwischen den Kontakten 13 und 4 sowie 6 und 8.
Bei der zweiten Messung wurde die Hallspannung~$U_{xy}$ an den
Kontakten 13 und 8 gemessen und die Längsspannung~$U_{xx}$ an 6 und 7.

In Abbildung \ref{fig:magnetotransport_led} sind die
entsprechenden Hall- und Längswiderstände $R_{xy}$ und $R_{xx}$
zu den Messwerten der Spannungen vom Hochfahren des Magnetfeldes dargestellt.

\begin{figure}[h!]
  \begin{center}
      \input{Magnetotransport_led}
  \end{center}
  \caption{
	Hall- und Längswiderstände an Probe 4 (kein Topgate)
	während \SI{10}{\micro{}A} zwischen den Kontakten 5 und 11 fließen.
	Die Probe wurde zuvor von einer LED bestrahlt.
	Die eingezeichneten Füllfaktoren sind rein rechnerisch bestimmt.}
  \label{fig:magnetotransport_led}
\end{figure}


\subsubsection*{Elektronenkonzentration}
Die zu erwartende Erhöhung der Elektronenkonzentration wird wie zuvor
über die Steigung des klassischen Halleffektes mit
Formel~\eqref{eq:ele_konz_steigung} berechnet.

In dem Bereich des klassischen Halleffektes
(siehe Abbildung~\ref{fig:magnetotransport_led}) beträgt die Steigung:
\begin{equation*}
\frac{\partial R_{xy}}{\partial B} = \SI{1183,47}{\Omega/T}
\end{equation*}
Woraus sich folgende Elektronenkonzentration berechnet:
\begin{equation*} n_e = \SI{5,2744e15}{m^{-2}} \end{equation*}

Das entspricht einer Erhöhung von \SI{6,517e14}{m^{-2}} (14\%).

\subsubsection*{Elektronenbeweglichkeit}

Die veränderte Elektronenkonzentration beeinflusst 
die Elektronenbeweglichkeit nach \eqref{eq:elektronenbeweglichkeit} 
zwischen den Kontakten 6 und 7 wie folgt:
\begin{equation*}
\mu_e = \frac{ \SI{275}{\mu m} }
 	     {\SI{69,9723}{\Omega} \cdot \SI{200}{\mu m} \cdot e \cdot \SI{5,2744e15}{m^{-2}}}
      = \SI{23.256}{m^2 / Vs}
\end{equation*}

%TODO: Hat sich erhöht, sollte sich aber vermindern (?)

\newpage

\section*{Messungen mit Topgate-Spannung}

Im Folgenden wird eine Magnetotransportmessung mit Probe~5
durchgeführt. Diese besitzt an Kontakt~13 einen sogenannten Topgate-Kontakt.
Eine angelegte Topgate-Spannung verdrängt Elektronen und vermindert so
die Elektronenkonzentration.
Es wird mit einer Topgate-Spannung von \SI{0}{mV}--\SI{200}{mV} gearbeitet.
Die Ergebnisse finden sich in Abbildung~\ref{fig:magnetotransport_topgate}.

\begin{figure}[h!]
  \begin{center}
    \input{Magnetotransport_topgate}
  \end{center}
  \caption{
	Hall- (oben) und Längswiderstände (unten) an Probe 5 (mit Topgate)
	während \SI{10}{\micro{}A} zwischen den Kontakten 5 und 11 fließen
	und eine Topgate-Spannung (Legende) an Kontakt~13 anliegt.
	Die eingezeichneten Füllfaktoren sind rein rechnerisch bestimmt.}
  \label{fig:magnetotransport_topgate}
\end{figure}

Man erkennt, dass die Steigung im Graphen der Hallspannung mit steigender
Topgate-Spannung zunimmt.
Die Frequenzen der Schubnikow-de-Haas-Oszillationen nehmen ebenfalls zu,
sowie deren Amplitude.

Ab einer Topgate-Spannung von \SI{200}{mV} fällt der Längswiderstand~$R_xx$
nicht mehr ab. Offenbar wurden fast alle Elektronen aus der Probe verdrängt.

Eine Lücke im Graphen für die \SI{0}{mV}-Topgate-Spannung bei \SI{3,5}{T}
entstand aufgrund eines losen Steckers.



\subsubsection*{Veränderung der Elektronenkonzentration}
Die angesprochene Verdrängung der Elektronen und die damit veränderte
Elektronenkonzentration wird nun mit Hilfe der $B^{-1}$-Oszillation
in Tabelle~\ref{tbl:ele_konz_topgate} berechnet und in
Abbildung~\ref{fig:ele_konz_topgate} dargestellt.

\begin{table}[h!]
\begin{center}
\begin{tabular}{rcc}
\toprule
$\frac{U_{\text{top}}}{\SI{}{mV}}$ &
$\frac{\Delta B^{-1}}{\SI{}{1/T}}$ &
$\frac{n_e}{\SI{e15}{m^{-2}}}$ \\
\midrule
0   & \num{0.0998} & 4,8451 \\
10  & \num{0.1022} & 4,7314 \\
20  & \num{0.1028} & 4,7037 \\
30  & \num{0.1046} & 4,6228 \\
100 & \num{0.1089} & 4,4403 \\
150 & \num{0.1216} & 3,9765 \\
200 & \num{0.1652} & 2,9271 \\
\bottomrule
\end{tabular}
\end{center}
\caption{
	Nach Formel~\eqref{eq:elektronen_osz} bestimmte Elektronenkonzentrationen 
	in Abhängigkeit der Topgate-Spannung}
\label{tbl:ele_konz_topgate}
\end{table}

\begin{figure}[h!]
  \begin{center}
    \input{Elektronenkonzentration_topgate}
  \end{center}
  \caption{
	Elektronenkonzentration~$n_e$ in Abhängigkeit der
	Topgate-Spannung~$U_\text{top}$ nach Tabelle~\ref{tbl:ele_konz_topgate}}
  \label{fig:ele_konz_topgate}
\end{figure}


Die Elektronenkonzentration nimmt offenbar linear mit der Topgate-Spannung
ab und zwar nach
\begin{equation*}
	n_e = \SI{4.9113e-5}{m^{-2}} - \SI{0.008}{m^{-2}V^{-1}} \cdot U_\text{top}
\end{equation*}




%TODO: Quellen-Versicherung
\newpage
\bibliography{Auswertung}{}
\end{document}
