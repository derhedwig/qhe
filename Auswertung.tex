% Copyright 2009 by Henrik Kroeger
%
% This file may be distributed and/or modified
% under the GNU Public License.

\documentclass[a4paper]{article}
\usepackage[ngerman]{babel}
\usepackage[T1]{fontenc}
\usepackage[utf8]{inputenc}

\usepackage{hyperref}

%\usepackage{parskip}

\usepackage{booktabs}
\usepackage{cellspace}
\usepackage{dcolumn}

\usepackage[amssymb]{SIunits}
\usepackage{sistyle}
\SIstyle{German}

\usepackage{multicol}

\usepackage{xcolor}
\usepackage{graphicx}
\usepackage{rotating}
\usepackage[loose]{subfigure}

\usepackage{tikz}
\usepackage{pgffor}
\usepackage{pgfmath}

\begin{document}

\title{Versuche zum Quanten-Hall-Effekt}
\author{Sarah Paczkowski%\thanks{Matrikelnummer: 2616640}
 \and Henrik Kröger%\thanks{Matrikelnummer: 2619840}
}
\date{27. und 28. Oktober 2009}

\maketitle

\noindent
Der \emph{Hall-Effekt} ist das Entstehen einer Potentialdifferenz
(der \emph{Hall-Spannung} $U_H$)
in einem elektrischen Leiter senkrecht zu einem elektrischen Strom $I$ und
senkrecht zu einem Magnetfeld $B$.
Der \emph{Hall-Widerstand} $R_H = \frac{U_H}{I}$,
verhält sich beim gewöhnlichen Hall-Effekt wie folgt:
\begin{equation} R_H = A_H \frac{B}{d} \end{equation}
$A_H$ ist dabei die materialabhängige \emph{Hall-Konstante} und
$d$ die Dicke des Leiters in Richtung des Magnetfeldes.
Im Fall eines Leiters mit nur einer Ladungsträgerart errechnet sich
die Hall-Konstante aus der Ladungsträgerdichte $n_e$ und
der Ladung $q$ der Ladungsträger:
\begin{equation} A_H = \frac{1}{n_eq} \end{equation}

\begin{figure}[hb]
\begin{center}
\begin{tikzpicture}

  % Box
  \draw (0,0) rectangle (2,1);
  \draw (0.2,0.2) rectangle (2.2,1.2);
  \draw (0,0) -- (0.2,0.2);
  \draw (2,1) -- (2.2,1.2);
  \draw (2,0) -- (2.2,0.2);
  \draw (0,1) -- (0.2,1.2);

  % Strom I
  \draw[->] (-1,0.6) -- (0.1,0.6) node[above,near start] {$I$};
  \draw[->] (2.1,0.6) -- (3,0.6) node[above,near end] {$I$};

  % Megnetfeld B
  \draw[->,thin] (1.3,0.8) -- (0.8,0.3) node[right,midway] {$B$};

  % Breite d
  \draw (0,1.25) -- (0.2,1.45) node[right,above] {$d$};
  \draw[very thin] (0,1) -- (0,1.3);
  \draw[very thin] (0.2,1.2) -- (0.2,1.5);

  % Voltmeter U
  \draw (4,0.6) node [circle, draw] (Voltmeter) {V} node[right=4mm] {$U_H$};
  \draw [out=90,in=90] (1,1.1) to (Voltmeter);
  \draw [out=-90,in=-90] (1,0.1) to (Voltmeter);

\end{tikzpicture}
\end{center}
\caption{Zum klassischen Hall-Effekt}
\end{figure}

Bei hohen Magnetfeldern (mehrere Tesla) und
tiefen Temperaturen (wenige Kelvin)
setzt dagegen der \emph{Quanten-Hall-Effekt} ein.
Hier hängt die Hall-Spannung nicht mehr linear vom Magnetfeld ab,
sondern weist Stufen auf.
Der Hall-Widerstand kommt dabei in Brüchen
der \emph{von-Klitzing-Konstanten} $R_K$ vor:
\begin{equation} R_H = \frac{R_K}{\nu} \end{equation}
Dabei ist $\nu$ der sogenannte \emph{Füllfaktor}.
Bei ganzzahligen Füllfaktoren spricht man
vom \emph{integralen Quanten-Hall-Effekt},
bei gebrochen-rationalen
vom \emph{fraktionalen Quanten-Hall-Effekt}.
Für den letzteren braucht es noch tiefere Temperaturen
und eine geringere \emph{Elektronenkonzentration} $n_e$ im Leiter.

Die von-Klitzing-Konstante lässt sich aus Naturkonstanten errechnen:
\begin{equation}
R_K = \frac{h}{e^2} \approx \SI{25,812807}{k\Omega}
\end{equation}


\section*{Proben}

Als Leiter für die Versuche werden Aluminium-Gallium-Arsenid-Heterostrukturen verwendet.


\begin{figure}[hb]
\begin{center}
\scriptsize
\subfigure[Probe 4]{
\begin{tikzpicture}[%
scale=0.2,
inner/.style={rectangle,draw,minimum size=2mm},
outer/.style={rectangle,draw}
]
  % Innere Kontakte
  \draw (0,0) node[inner] (inner 11) {};
  \draw (0,-9) node[inner] (inner 5) {};

  \draw (-1.5,-3) node[inner] (inner 8) {};
  \draw (-1.5,-4.5) node[inner] (inner 7) {};
  \draw (-1.5,-6) node[inner] (inner 6) {};
  \draw (-1.5,-7.5) node[inner] {};

  \draw (1.5,-1.5) node[inner] {};
  \draw (1.5,-3) node[inner] (inner 13) {};
  \draw (1.5,-4.5) node[inner] (inner 19) {};
  \draw (1.5,-6) node[inner] (inner 4) {};

  % Carrier
%  \draw (-5,-13) rectangle +(17,17);
%  \draw (-7,-15) rectangle +(21,21);
  \foreach \x in {1,...,5}
    \pgfmathparse{3*(5-\x)-1.75}
    \draw (\pgfmathresult,-15) node[outer,anchor=south east,minimum height=0.4cm] (outer \x) {0\x};
  \foreach \x in {11,...,15}
    \pgfmathparse{3*(\x-11)-1.75}
    \draw (\pgfmathresult,4) node[outer,anchor=south east,minimum height=0.4cm] (outer \x) {\x};
  \foreach \x in {6,...,9}
    \pgfmathparse{2*(\x-6)-9}
    \draw (-5,\pgfmathresult) node[outer,anchor=south east,minimum width=0.4cm] (outer \x) {0\x};
  \draw (-5,-1) node[outer,anchor=south east,minimum width=0.4cm] (outer 10) {10};
  \foreach \x in {16,...,20}
    \pgfmathparse{2*(16-\x)-1}
    \draw (14,\pgfmathresult) node[outer,anchor=south east,minimum width=0.4cm] (outer \x) {\x};

  % Verbindungen
  \draw[in=-90,out=90] (outer 5) to (inner 5);
  \draw[in=0,out=90] (outer 4) to (inner 4);
  \draw[in=90,out=-90] (outer 11) to (inner 11);
  \draw[in=0,out=-90] (outer 13) to (inner 13);
  \draw[in=180,out=0] (outer 8) to (inner 8);
  \draw[in=180,out=0] (outer 7) to (inner 7);
  \draw[in=180,out=0] (outer 6) to (inner 6);
  \draw[in=0,out=180] (outer 19) to (inner 19);

\end{tikzpicture}
}
\qquad
\subfigure[Probe 5 mit Topgate]{
\begin{tikzpicture}[%
scale=0.2,
inner/.style={rectangle,draw,minimum size=2mm},
outer/.style={rectangle,draw}
]

  % Innere Kontakte
  \draw (0,0) node[inner] (inner 12) {};
  \draw (0,-9) node[inner] (inner 4) {};

  \draw (-1.5,-1.5) node[inner] {};
  \draw (-1.5,-3) node[inner] (inner 9) {};
  \draw (-1.5,-4.5) node[inner] (inner 8) {};
  \draw (-1.5,-6) node[inner] (inner 7) {};
  \draw (-1.5,-7.5) node[inner] (inner 5) {};

  \draw (1.5,-1.5) node[inner] (inner 13) {};
  \draw (1.5,-3) node[inner] (inner 17) {};
  \draw (1.5,-4.5) node[inner] (inner 18) {};
  \draw (1.5,-6) node[inner] (inner 3) {};

  % Carrier
%  \draw (-5,-13) rectangle +(17,17);
%  \draw (-7,-15) rectangle +(21,21);
  \foreach \x in {1,...,5}
    \pgfmathparse{3*(5-\x)-1.75}
    \draw (\pgfmathresult,-15) node[outer,anchor=south east,minimum height=0.4cm] (outer \x) {0\x};
  \foreach \x in {11,...,15}
    \pgfmathparse{3*(\x-11)-1.75}
    \draw (\pgfmathresult,4) node[outer,anchor=south east,minimum height=0.4cm] (outer \x) {\x};
  \foreach \x in {6,...,9}
    \pgfmathparse{2*(\x-6)-9}
    \draw (-5,\pgfmathresult) node[outer,anchor=south east,minimum width=0.4cm] (outer \x) {0\x};
  \draw (-5,-1) node[outer,anchor=south east,minimum width=0.4cm] (outer 10) {10};
  \foreach \x in {16,...,20}
    \pgfmathparse{2*(16-\x)-1}
    \draw (14,\pgfmathresult) node[outer,anchor=south east,minimum width=0.4cm] (outer \x) {\x};

  % Verbindungen
  \draw[in=180,out=0] (outer 7) to (inner 7);
  \draw[in=180,out=0] (outer 8) to (inner 8);
  \draw[in=180,out=0] (outer 9) to (inner 9);
  \draw[in=0,out=180] (outer 17) to (inner 17);
  \draw[in=0,out=180] (outer 18) to (inner 18);
  \draw[in=90,out=-90] (outer 12) to (inner 12);
  \draw[in=0,out=-90] (outer 13) to (inner 13);
  \draw[in=0,out=90] (outer 3) to (inner 3);
  \draw[in=-90,out=90] (outer 4) to (inner 4);
  \draw[in=180,out=90] (outer 5) to (inner 5);

\end{tikzpicture}
}
\end{center}
\normalsize
\caption{Bondschemata}
\end{figure}

\end{document}
